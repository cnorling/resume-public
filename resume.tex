% class
% ----------------------------------------------------------------------
\documentclass{resume}

% inputs
% variables in LaTeX are declared with \newcommand for some reason
% ----------------------------------------------------------------------
\newcommand{\name}    {john johnersen}      % enter your name
\newcommand{\github}  {username}            % enter your github username
\newcommand{\website} {google.com}          % enter your website
\newcommand{\email}   {foo@gmail.com}       % enter your email address
\newcommand{\phone}   {555-555-5555}        % enter your phone numberl
\newcommand{\address} {55555}               % enter your zip code
\newcommand{\colour}  {Bittersweet}         % pick a primary color from this list https://sharelatex-wiki-cdn-671420.c.cdn77.org/learn-scripts/images/e/ef/OLxcolorList2.png

% document
% ----------------------------------------------------------------------
\begin{document}
\heading

\sector{overview}
I'm an IT professional looking for work in the tech industry. I have a passionate love of all things technology. I'm capable of doing things like working with infrastructure, maintaining codebases, and provisioning servers. I love to solve complex problems for organizations.

% use sectors to declare large sections
\sector{qualifications}

% use qualifications to create a list of items that spans the whole page
\begin{qualifications}
    \vspace*{5pt}
    \item Lorem ipsum dolor sit amet, consectetuer adipiscing elit. Ut purus elit, vestibulum
    \item placerat ac, adipiscing vitae, felis. Curabitur dictum gravida mauris. Nam arcu libero
    \item nonummy eget, consectetuer id, vulputate a, magna. Donec vehicula augue eu neque
    \item Pellentesque habitant morbi tristique senectus et netus et malesuada fames ac turpis
    \item Aenean faucibus. Morbi dolor nulla, malesuada eu, pulvinar at, mollis ac, nulla. Curabitur
    \item egestas. Mauris ut leo. Cras viverra metus rhoncus sem. Nulla et lectus vestibulum urna fringilla ultrices. Phasellus eu tellus sit amet tortor gravida placerat. Integer sapien est,
    % \item comment out lines that aren't relevant to the job you're applying to
    \vspace*{5pt}
\end{qualifications}

\sector{skills \& tools}

% use skills to create a list of items with multiple columns
% the number denotes how many columns there should be.
% I wouldn't recommend going over five
\begin{skills}{4}
    \item foo
    \item fom
    \item fol
    \item foz
    \item foe
    \item loe
    \item lom
    \item lin
    \item loz
    \item riz
    \item roz
    \item ruz
    \item rux
    \item roe
    \item rum
    \item rul
    \item rie
    \item rip
    \item vil
    \item vom
    % \item vul         comment out skills that aren't relevant to the job
    % \item bar         and they'll drop out of the list
    % \item baz
    % \item buz
\end{skills}

\sector{experience}

\begin{job}{Job Name}{Job Title}{jan 2022 - jan 2023}
    \lipsum[1-2]
    \begin{qualifications}
        \item Lorem ipsum dolor sit amet, consectetuer adipiscing elit. Ut purus elit, vestibulum
        \item placerat ac, adipiscing vitae, felis. Curabitur dictum gravida mauris. Nam arcu libero
        \item nonummy eget, consectetuer id, vulputate a, magna. Donec vehicula augue eu neque
    \end{qualifications}    
\end{job}

% use \nextpage to insert a clean break onto the next page 
% includes a footer indicating to continue reading
\nextpage

\begin{job}{Job Name}{Job Title}{jan 2021 - jan 2022}
    \lipsum[2]
\end{job}

\begin{job}{Job Name}{Job Title}{jan 2020 - jan 2021}
    \lipsum[7]
\end{job}

\begin{job}{Job Name}{Job Title}{jan 2019 - jan 2020}
    \lipsum[2]
\end{job}

\begin{job}{Job Name}{Job Title}{jan 2018 - jan 2019}
    \lipsum[7]
    \begin{qualifications}
        \item Lorem ipsum dolor sit amet, consectetuer adipiscing elit. Ut purus elit, vestibulum
        \item placerat ac, adipiscing vitae, felis. Curabitur dictum gravida mauris. Nam arcu libero
        \item nonummy eget, consectetuer id, vulputate a, magna. Donec vehicula augue eu neque
    \end{qualifications}    
\end{job}

\end{document}
